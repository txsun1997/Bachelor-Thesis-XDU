
\begin{acknowledgments}

  四载匆匆,倏忽而至,终于要毕业了。
  
  我还清楚地记得那段穿着军装、踢着正步、烈日永远不会缺席的时光,记得敲下第一行\texttt{Hello World}时的懵懂,也还记得和队友们一起通宵做数模的疲惫,更记得无数个平凡的日子里坐在教室里一起听课的同学们和站在讲台上的教授们……一切仿佛就在昨日,如今却要离开这里了。
  
  回首这段岁月,我只觉得幸运。何其幸运,能遇到这么多学识渊博、兢兢业业的老师;何其幸运,能遇到这么多优秀又谦虚的同学们。借此机会,我想向你们致以真挚的感谢。
  
  首先,我想感谢本文的指导老师,也即将是我博士阶段的导师:邱锡鹏老师和黄萱菁老师。没有他们提供的悉心指导和实验环境,就不会有这篇文章。另外,还要特别感谢邱老师,他的学术品位、研究热情和为人方式都激励着我,我相信我也将度过同样充实且舒适的博士生涯。
  
  然后,我要感谢本科阶段给予我巨大帮助的老师们和同学们。感谢陈慧婵老师和郭艳艳老师,他们为我打下了扎实的数理基础;感谢张淑平老师、张立勇老师和顾新老师,让我掌握了必要的专业知识;感谢周水生老师,在我参与数模期间提供了大量指导。感谢一路同行的同学们,王磊、余天焕、徐之浩、班浩等等,你们一直是我成长路上的榜样;还要感谢陪伴我四年的室友,王敏锐、王许丞和冯尧,感谢你们一直以来的包容与帮助。
  
  最后,我要感谢我的家人。感谢我的父母,在我的求学路上面临的每一次选择,你们都给予了最大的鼓励与信任,没有你们一如既往的支持,也不会有现在的我;还要感谢我的女朋友曲雪纯,遇到你是我本科阶段最大的收获之一。你们一直,也将永远是我的最强大的后盾!
  
  \begin{flushright}
  	\emph{孙天祥}\\
  	二零一九年六月于西电
  \end{flushright}

\end{acknowledgments}

